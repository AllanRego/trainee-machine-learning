\documentclass{beamer}

\usepackage[utf8]{inputenc}

\title{Visão Computacional}
\subtitle{Introdução ao tópico}
\author{Vitor Greati\inst{1} \and Vinícius Campos\inst{1}}
\institute[]
{
	\inst{1}%
	Universidade Federal do Rio Grande do Norte
}
\date{}
\subject{Computer Science}

% Table of contents at the beginning of each section
\AtBeginSection[]
{
  \begin{frame}
    \frametitle{Sumário}
    \tableofcontents[currentsection, currentsubsection]
  \end{frame}
}

% Table of contents at the beginning of each subsection
%\AtBeginSubsection[]
%{
%  \begin{frame}
%    \frametitle{Table of Contents}
%    \tableofcontents[currentsection,currentsubsection]
%  \end{frame}
%}

\begin{document}

\frame{\titlepage}

\section{O gap semântico}

    \begin{frame}{O que você percebe nesta imagem?}

    \end{frame}

    \begin{frame}{A percepção do computador}

    \end{frame}
    
    \begin{frame}{O \emph{gap} semântico}{Percepção humana vs. Percepção da máquina}
    
    \end{frame}

\section{Visão Computacional}

    \subsection{Para além de \emph{pixels}}

    \begin{frame}{O \emph{gap} semântico}{Percepção humana vs. Percepção da máquina}
        Capturar fatos do mundo por uma ou mais imagens e realizar o inverso: reconstruir suas propriedades, 
        como formas, iluminação e distribuições de cor.
    \end{frame}

    \subsection{Aplicações}

    \subsection{Técnicas}

\section{Aprendizagem de Máquina}


\end{document}
